%!TEX root = std.tex
\rSec0[cpp]{预处理指令}%
\indextext{preprocessing directives|(}

\indextext{compiler control line|see{preprocessing directives}}%
\indextext{control line|see{preprocessing directives}}%
\indextext{directive, preprocessing|see{preprocessing directives}}

%gram: \rSec1[gram.cpp]{Preprocessing directives}
%gram:

\pnum
\defn{预处理指令}是满足下列约束的预处理符号序列:
序列第一个符号是预处理符号\tcode{\#}
并且是源文件的第一个字符(前方可以存在不含换行符的空白)
或位于至少含有一个换行符的空白后。
序列最后一个符号是序列第一个符号之后遇到的第一个换行符。
\footnote{因此,
	预处理指令通常称作``行''。
	``行''没有另外的语法含义,
	因为除了预处理期间某些情况较特殊之外,
	其它时候所有的空白都是等价的
	(例如,\ref{cpp.stringize}中的字符字串创建符\tcode{\#})。}
预处理指令中的函数风格宏调用不能拆分在多行,
换行符必定意味着预处理指令结束。

\begin{bnf}
\nontermdef{预处理文件}\br
    组\opt
\end{bnf}

\begin{bnf}
\nontermdef{组}\br
    组分\br
    组~组分
\end{bnf}

\begin{bnf}
\nontermdef{组分}\br
    判断段\br
    控制行\br
    文本行\br
    \terminal{\#} 条件支持指令
\end{bnf}

\begin{bnf}
\nontermdef{判断段}\br
    if组~elif组序列\opt~else组\opt~判断结束行
\end{bnf}

\begin{bnftab}
\nontermdef{if组}\br
\>\terminal{\# if}\>\>常量表达式~换行~组\opt\br
\>\terminal{\# ifdef}\>\>标识符~换行~组\opt\br
\>\terminal{\# ifndef}\>\>标识符~换行~组\opt
\end{bnftab}

\begin{bnf}
\nontermdef{elif组序列}\br
    elif组\br
    elif组序列~elif组
\end{bnf}

\begin{bnftab}
\nontermdef{elif组}\br
\>\terminal{\# elif}\>\>常量表达式~换行~组\opt
\end{bnftab}

\begin{bnftab}
\nontermdef{else组}\br
\>\terminal{\# else}\>\>换行~组\opt
\end{bnftab}

\begin{bnftab}
\nontermdef{判断结束行}\br
\>\terminal{\# endif}\>\>换行
\end{bnftab}

\begin{bnftab}
\nontermdef{控制行}\br
\>\terminal{\# include}\>\>预处理符号序列~换行\br
\>\terminal{\# define}\>\>标识符~替换列表~换行\br
\>\terminal{\# define}\>\>标识符~左括弧~标识符列表\opt{}~\terminal{)}~替换列表~换行\br
\>\terminal{\# define}\>\>标识符~左括弧~\terminal{... )}~替换列表~换行\br
\>\terminal{\# define}\>\>标识符~左括弧~标识符列表~,~\terminal{... )}~替换列表~换行\br
\>\terminal{\# undef}\>\>标识符~换行\br
\>\terminal{\# line}\>\>预处理符号序列~换行\br
\>\terminal{\# error}\>\>预处理符号序列\opt~换行\br
\>\terminal{\# pragma}\>\>预处理符号序列\opt~换行\br
\>\terminal{\# }换行
\end{bnftab}

\begin{bnf}
\nontermdef{文本行}\br
    预处理符号序列\opt~换行
\end{bnf}

\begin{bnf}
\nontermdef{条件支持指令}\br
    预处理符号序列~换行
\end{bnf}

\begin{bnf}
\nontermdef{左括弧}\br
    \descr{不紧跟空白之后的\terminal{(}字符}
\end{bnf}

\begin{bnf}
\nontermdef{标识符列表}\br
    标识符\br
    标识符列表~\terminal{,}~标识符
\end{bnf}

\begin{bnf}
\nontermdef{替换列表}\br
    预处理符号序列\opt
\end{bnf}

\begin{bnf}
\nontermdef{预处理符号序列}\br
    预处理符号\br
    预处理符号序列~预处理符号
\end{bnf}

\begin{bnf}
\nontermdef{换行}\br
    \descr{换行符}
\end{bnf}

\pnum
文本行不以预处理符号\tcode{\#}开头。
\grammarterm{条件支持指令}不以上列语法中的指令名开头。
\grammarterm{条件支持指令}是条件支持的,
其语义是\impldef{additional supported forms of preprocessing directive}。

\pnum
当跳过一个组时~(\ref{cpp.cond}),
组中指令的语法不再受限,指令名和随后的换行符之间
可以是任意预处理符号。

\pnum
预处理指令中
(从开头的预处理符号\tcode{\#}到结尾的换行符之间)
仅允许出现空格和水平制表这两种空白
(包括已在翻译第三阶段被替换为空白的注释以及其它可能的空白符)。

\pnum
实现可以条件地处理或跳过源文件的某些部分、
包含其它的源文件以及进行宏替换。
由于从概念上来说这些行为都发生在翻译最后的翻译单元之前,
因此这些功能称为\term{预处理}。

\pnum
若无特殊说明,预处理指令中的预处理符号不作宏展开。

\enterexample 在:

\begin{codeblock}
#define EMPTY
EMPTY   #   include <file.h>
\end{codeblock}

中,虽然宏\tcode{EMPTY}被替换后第二行看似成为预处理指令,
但第二行的预处理符号序列\textit{不构成}预处理指令,
因为该序列在翻译第四阶段未以\#开始。

\rSec1[cpp.cond]{条件包含}%
\indextext{preprocessing directive!conditional inclusion}%
\indextext{inclusion!conditional|see{preprocessing directive, conditional inclusion}}

\indextext{\idxcode{defined}}%
\begin{ncbnf}
\nontermdef{defined宏表达式}\br
    \terminal{defined} 标识符\br
    \terminal{defined (} 标识符 \terminal{)}
\end{ncbnf}

\begin{ncbnf}
\nontermdef{h-预处理符号}\br
    \textnormal{除\terminal{>}以外的任何\nonterminal{预处理符号}}
\end{ncbnf}

\begin{ncbnf}
\nontermdef{h-预处理符号序列}\br
    h-预处理符号\br
    h-预处理符号序列 h-预处理符号
\end{ncbnf}

\indextext{\idxcode{__has_include}}%
\begin{ncbnf}
\nontermdef{has-include表达式}\br
    \terminal{__has_include ( <} h-字符序列 \terminal{> )}\br
    \terminal{__has_include ( "} q-字符序列 \terminal{" )}\br
    \terminal{__has_include (}   字符串字面值  \terminal{)}\br
    \terminal{__has_include ( <} h-预处理符号序列     \terminal{> )}
\end{ncbnf}

\pnum
The expression that controls conditional inclusion
shall be an integral constant expression except that
identifiers
(including those lexically identical to keywords)
are interpreted as described below\footnote{Because the controlling constant expression is evaluated
during translation phase 4,
all identifiers either are or are not macro names ---
there simply are no keywords, enumeration constants, etc.}
and it may contain zero or more \grammarterm{defined-macro-expression}{s} and/or
\grammarterm{has-include-expression}{s} as unary operator expressions.

\pnum
A \grammarterm{defined-macro-expression} evaluates to \tcode{1}
if the identifier is currently defined
as a macro name
(that is, if it is predefined
or if it has been the subject of a
\tcode{\#define}
preprocessing directive
without an intervening
\tcode{\#undef}
directive with the same subject identifier), \tcode{0} if it is not.

\pnum
The third and fourth forms of \grammarterm{has-include-expression}
are considered only if neither of the first or second forms matches,
in which case the preprocessing tokens are processed just as in normal text.

\pnum
The header or source file identified by
the parenthesized preprocessing token sequence
in each contained \grammarterm{has-include-expression}
is searched for as if that preprocessing token sequence
were the \grammarterm{pp-tokens} in a \tcode{\#include} directive,
except that no further macro expansion is performed.
If such a directive would not satisfy the syntactic requirements
of a \tcode{\#include} directive, the program is ill-formed.
The \grammarterm{has-include-expression} evaluates
to \tcode{1} if the search for the source file succeeds, and
to \tcode{0} if the search fails.

\pnum
The \tcode{\#ifdef} and \tcode{\#ifndef} directives, and
the \tcode{defined} conditional inclusion operator,
shall treat \tcode{__has_include} as if it were the name of a defined macro.
The identifier \tcode{__has_include} shall not appear
in any context not mentioned in this section.

\pnum
Each preprocessing token that remains (in the list of preprocessing tokens that
will become the controlling expression)
after all macro replacements have occurred
shall be in the lexical form of a token~(\ref{lex.token}).

\pnum
Preprocessing directives of the forms

\begin{ncbnftab}
\indextext{\idxcode{\#if}}%
\terminal{\# if}\>\>constant-expression new-line group\opt\br
\indextext{\idxcode{\#elif}}%
\terminal{\# elif}\>\>constant-expression new-line group\opt
\end{ncbnftab}

check whether the controlling constant expression evaluates to nonzero.

\pnum
Prior to evaluation,
macro invocations in the list of preprocessing tokens
that will become the controlling constant expression
are replaced
(except for those macro names modified by the
\tcode{defined}
unary operator),
just as in normal text.
If the token
\tcode{defined}
is generated as a result of this replacement process
or use of the
\tcode{defined}
unary operator does not match one of the two specified forms
prior to macro replacement,
the behavior is undefined.
After all replacements due to macro expansion and
evaluations of \grammarterm{defined-macro-expression}{s} and
\grammarterm{has-include-expression}{s}
have been performed,
all remaining identifiers and keywords\footnote{An alternative
token~(\ref{lex.digraph}) is not an identifier,
even when its spelling consists entirely of letters and underscores.
Therefore it is not subject to this replacement.},
except for
\tcode{true}
and
\tcode{false},
are replaced with the pp-number
\tcode{0},
and then each preprocessing token is converted into a token.
The resulting tokens comprise the controlling constant expression
which is evaluated according to the rules of~\ref{expr.const}
using arithmetic that has at least the ranges specified
in~\ref{support.limits}. For the purposes of this token conversion and evaluation
all signed and unsigned integer types
act as if they have the same representation as, respectively,
\tcode{intmax_t} or \tcode{uintmax_t}~(\ref{cstdint}).\footnote{Thus on an
implementation where \tcode{std::numeric_limits<int>::max()} is \tcode{0x7FFF}
and \tcode{std::numeric_limits<unsigned int>::max()} is \tcode{0xFFFF},
the integer literal \tcode{0x8000} is signed and positive within a \tcode{\#if}
expression even though it is unsigned in translation phase
7~(\ref{lex.phases}).}
This includes interpreting character literals, which may involve
converting escape sequences into execution character set members.
Whether the numeric value for these character literals
matches the value obtained when an identical character literal
occurs in an expression
(other than within a
\tcode{\#if}
or
\tcode{\#elif}
directive)
is \impldef{numeric values of character literals in \tcode{\#if}
directives}.\footnote{Thus, the constant expression in the following
\tcode{\#if}
directive and
\tcode{if}
statement is not guaranteed to evaluate to the same value in these two
contexts.
\begin{tabbing}
\hspace{.6in}\=\kill%
\>\tcode{\#if 'z' - 'a' == 25}\\
\>\tcode{if ('z' - 'a' == 25)}
\end{tabbing}
}
Also, whether a single-character character literal may have a negative
value is \impldef{negative value of character literal in preprocessor}.
Each subexpression with type
\tcode{bool}
is subjected to integral promotion before processing continues.

\pnum
Preprocessing directives of the forms

\begin{ncbnftab}
\terminal{\# ifdef}\>\>identifier new-line group\opt\br
\indextext{\idxcode{\#ifdef}}%
\terminal{\# ifndef}\>\>identifier new-line group\opt
\indextext{\idxcode{\#ifndef}}%
\end{ncbnftab}

check whether the identifier is or is not currently defined as a macro name.
Their conditions are equivalent to
\tcode{\#if}
\tcode{defined}
\term{identifier}
and
\tcode{\#if}
\tcode{!defined}
\term{identifier}
respectively.

\pnum
Each directive's condition is checked in order.
If it evaluates to false (zero),
the group that it controls is skipped:
directives are processed only through the name that determines
the directive in order to keep track of the level
of nested conditionals;
the rest of the directives' preprocessing tokens are ignored,
as are the other preprocessing tokens in the group.
Only the first group
whose control condition evaluates to true (nonzero) is processed;
any following groups are skipped and their controlling directives
are processed as if they were in a group that is skipped.
If none of the conditions evaluates to true,
and there is a
\tcode{\#else}
\indextext{\idxcode{\#else}}%
directive,
the group controlled by the
\tcode{\#else}
is processed; lacking a
\tcode{\#else}
directive, all the groups until the
\tcode{\#endif}
\indextext{\idxcode{\#endif}}%
are skipped.\footnote{As indicated by the syntax,
a preprocessing token shall not follow a
\tcode{\#else}
or
\tcode{\#endif}
directive before the terminating new-line character.
However,
comments may appear anywhere in a source file,
including within a preprocessing directive.}

\enterexample
This demonstrates a way to include a library \tcode{optional} facility
only if it is available:

\begin{codeblock}
#if __has_include(<optional>)
#  include <optional>
#  define have_optional 1
#elif __has_include(<experimental/optional>)
#  include <experimental/optional>
#  define have_optional 1
#  define experimental_optional 1
#else
#  define have_optional 0
#endif
\end{codeblock}
\exitexample

\rSec1[cpp.include]{Source file inclusion}
\indextext{preprocessing directives!header inclusion}
\indextext{preprocessing directives!source-file inclusion}
\indextext{inclusion!source~file|see{preprocessing directives, source-file inclusion}}%
\indextext{\idxcode{\#include}}%

\pnum
A
\tcode{\#include}
directive shall identify a header or source file
that can be processed by the implementation.

\pnum
\indextext{\idxcode{<...>}|see{preprocessing directive, header}}%
A preprocessing directive of the form

\begin{ncsimplebnf}
\terminal{\# include <}h-char-sequence\terminal{>} new-line
\end{ncsimplebnf}

searches a sequence of
\impldef{sequence of places searched for a header}
places
for a header identified uniquely by the specified sequence
between the
\tcode{<}
and
\tcode{>}
delimiters,
and causes the replacement of that
directive by the entire contents of the header.
How the places are specified
or the header identified
is \impldef{search locations for \tcode{<>} header}.

\pnum
\indextext{\idxcode{\"{}...\"{}}|see{preprocessing directives, source-file inclusion}}%
A preprocessing directive of the form

\begin{ncsimplebnf}
\terminal{\# include "}q-char-sequence\terminal{"} new-line
\end{ncsimplebnf}

causes the replacement of that
directive by the entire contents of the
source file identified by the specified sequence between the
\tcode{"}
delimiters.
The named source file is searched for in an
\impldef{manner of search for included source file}
manner.
If this search is not supported,
or if the search fails,
the directive is reprocessed as if it read

\begin{ncsimplebnf}
\terminal{\# include <}h-char-sequence\terminal{>} new-line
\end{ncsimplebnf}

with the identical contained sequence (including
\tcode{>}
characters, if any) from the original directive.

\pnum
A preprocessing directive of the form

\begin{ncsimplebnf}
\terminal{\# include} pp-tokens new-line
\end{ncsimplebnf}

(that does not match one of the two previous forms) is permitted.
The preprocessing tokens after
\tcode{include}
in the directive are processed just as in normal text
(i.e., each identifier currently defined as a macro name is replaced by its
replacement list of preprocessing tokens).
If the directive resulting after all replacements does not match
one of the two previous forms, the behavior is
undefined.\footnote{Note that adjacent string literals are not concatenated into
a single string literal
(see the translation phases in~\ref{lex.phases});
thus, an expansion that results in two string literals is an
invalid directive.}
The method by which a sequence of preprocessing tokens between a
\tcode{<}
and a
\tcode{>}
preprocessing token pair or a pair of
\tcode{"}
characters is combined into a single header name
preprocessing token is \impldef{search locations for \tcode{""""} header}.

\pnum
The implementation shall provide unique mappings for
sequences consisting of one or more
\grammarterm{nondigit}{s} or \grammarterm{digit}{s}~(\ref{lex.name})
followed by a period
(\tcode{.})
and a single
\grammarterm{nondigit}.
The first character shall not be a \grammarterm{digit}.
The implementation may ignore distinctions of alphabetical case.

\pnum
A
\tcode{\#include}
preprocessing directive may appear
in a source file that has been read because of a
\tcode{\#include}
directive in another file,
up to an \impldef{nesting limit for \tcode{\#include} directives} nesting limit.

\pnum
\enternote
Although an implementation may provide a mechanism for making arbitrary
source files available to the \tcode{< >} search, in general
programmers should use the \tcode{< >} form for headers provided
with the implementation, and the \tcode{" "} form for sources
outside the control of the implementation. For instance:

\begin{codeblock}
#include <stdio.h>
#include <unistd.h>
#include "usefullib.h"
#include "myprog.h"
\end{codeblock}

\exitnote

\pnum
\enterexample
This illustrates macro-replaced
\tcode{\#include}
directives:

\begin{codeblock}
#if VERSION == 1
    #define INCFILE  "vers1.h"
#elif VERSION == 2
    #define INCFILE  "vers2.h"   // and so on
#else
    #define INCFILE  "versN.h"
#endif
#include INCFILE
\end{codeblock}
\exitexample

\rSec1[cpp.replace]{宏替换}%
\indextext{macro!replacement|(}%
\indextext{replacement!macro|see{macro, replacement}}%
\indextext{preprocessing directives!macro replacement|see{macro, replacement}}

\pnum
\indextext{macro!replacement list}%
两个替换列表相同,
当且仅当它们的预处理符号的
数目、顺序、拼写和空白间隔都相同,
这里所有的空白间隔都视为相同。

\pnum
此时定义为
\indextext{object-like macro|see{macro, object-like}}%
\indextext{macro!object-like}%
\grammarterm{对象风格}
宏的标识符可以在彼时被其他\tcode{\#define}
预处理指令重定义,
只要彼时的定义也是对象风格宏定义
并且两次定义的替换列表相同,
如若不然程序是违规的。
同样地,
此时定义为
\indextext{function-like macro|see{macro, function-like}}%
\indextext{macro!function-like}%
\grammarterm{函数风格}
宏的标识符可以在彼时被其他
\tcode{\#define}
预处理指令重定义,
只要彼时的定义也是函数风格宏定义
并且两次定义具有相同数目和拼写的参数以及相同的替换列表,
如若不然程序是违规的。

\pnum
\indextext{macro!replacement list}%
在定义对象风格宏时,标识符和替换列表之间应留有空白。

\pnum
若定义宏时
\grammarterm{标识符列表}不以省略号结尾,
则在调用函数风格宏时实参的数目(包括那些零个预处理符号构成的实参)
应该等于宏定义中形参的数目;
否则,应该在调用时提供多于宏定义中(除\tcode{...}以外的)形参数目的实参。
必须以一个预处理符号\tcode{)}结束调用。

\pnum
\indextext{\xname{VA_ARGS}@\mname{VA_ARGS}}%
标识符\mname{VA_ARGS}仅应出现在形参中使用了省略号的函数风格宏的替换列表中。

\pnum
函数风格宏的形参标识符在其作用域范围内应该是独一无二的。

\pnum
紧跟\tcode{define}之后的标识符称为
\indextext{macro!name}%
\indextext{name!macro|see{macro, name}}%
\term{宏名}。
所有宏名都只共享一个命名空间。
不管是哪种风格的宏,
出现在替换列表前方或者后方的空白字符都不作为替换列表的一部分。

\pnum
若在可能是预处理指令开头的地方出现了预处理符号
\indextext{\#\#0~operator@\tcode{\#}~operator}
\tcode{\#}
后跟一个标识符,
该标识符不作宏替换。

\pnum
以下形式的预处理指令:

\begin{ncsimplebnf}
\terminal{\# define} 标识符~替换列表~换行
\indextext{\idxcode{\#define}}%
\end{ncsimplebnf}

定义了一个
\indextext{macro!object-like}%
\grammarterm{对象风格宏},
该宏名后续的所有实例
\footnote{由于宏替换时
	所有字符字面值和字符串字面值都是预处理符号
	而不是包含标识符的序列(见\ref{lex.phases},翻译阶段),
	因此它们不会被扫描宏名和形参时扫描。}
都会被替换为
构成指令剩余部分的预处理符号替换列表。
\footnote{替代符号(\ref{lex.digraph})不是标识符,即使它们
	的拼写全由字母和下划线构成。因此,无法定义和替代符号相同的宏名。}
接着该列表会再次扫描发现其它宏名,见下文。

\pnum
以下形式的预处理指令:

\begin{ncsimplebnf}
\terminal{\# define} 标识符~左括弧~标识符列表\opt{}~\terminal{)}~替换列表~换行\br
\terminal{\# define} 标识符~左括弧~\terminal{...}~\terminal{)}~替换列表~换行\br
\terminal{\# define} 标识符~左括弧~标识符列表~\terminal{, ...}~\terminal{)}~替换列表~换行\br
\end{ncsimplebnf}

\indextext{macro!function-like}%
定义了一个带有参数的\grammarterm{函数风格宏},
其用法在语法上很像函数调用。
参数
\indextext{parameters!macro}%
由可选的标识符列表指定,
其作用域由其在标识符列表中的声明开始,
到终止该\tcode{\#define}预处理指令的换行符处结束,
该宏名后续的所有实例后跟预处理符号\tcode{(}
都引入了一系列将要被定义中的替换列表进行替换的预处理符号
(宏调用)。
\indextext{invocation!macro}%
这些要进行替换的预处理符号序列由一个\tcode{)}终止,
注意这里不计序列间配成对的左右圆括号。
调用函数风格宏时,提供的预处理符号序列间出现的换行符都视为普通的空白符。

\pnum
\indextext{macro!function-like!arguments}%
最外层括号对括住的预处理符号序列
构成了函数风格宏的实参列表。
各个实参在列表中由逗号分隔,
但在内层括号对之间出现的逗号不属于分隔逗号。
If there are sequences of preprocessing tokens within the list of
arguments that would otherwise act as preprocessing directives,\footnote{Despite the name, a non-directive is a preprocessing directive.}
the behavior is undefined.

\pnum
\indextext{macro!function-like!arguments}%
If there is a \tcode{...} immediately preceding the \tcode{)} in the
function-like macro
definition, then the trailing arguments, including any separating comma preprocessing
tokens, are merged to form a single item: the \term{variable arguments}. The number of
arguments so combined is such that, following merger, the number of arguments is
one more than the number of parameters in the macro definition (excluding the
\tcode{...}).

\rSec2[cpp.subst]{参数替代}%
\indextext{macro!argument substitution}%
\indextext{argument~substitution|see{macro, argument substitution}}%

\pnum
一旦确定了调用函数式宏提供的实参,
参数替代就开始进行。
所有替换列表中前方无\tcode{\#}和\tcode{\#\#}、
或者后方无\tcode{\#\#}的形参(见下文)都会在实参作了宏展开后,
被相应的实参替代。
替代前,
每个实参的预处理符号序列都进行完全的宏替换,
其行为类似于它们形成了预处理文件的余下内容,
其它预处理符号都不再有效。

\pnum
After the arguments for the invocation of a function-like macro have
been identified, argument substitution takes place.
A parameter in the replacement list, unless preceded by a
\tcode{\#}
or
\tcode{\#\#}
preprocessing token or followed by a
\tcode{\#\#}
preprocessing token (see below),
is replaced by the corresponding argument after all macros
contained therein have been expanded.
Before being substituted,
each argument's preprocessing tokens are completely
macro replaced as if they formed the rest of the
preprocessing file;
no other preprocessing tokens are available.

\pnum
An identifier \mname{VA_ARGS} that occurs in the replacement list
shall be treated as if it were a parameter, and the variable arguments shall form
the preprocessing tokens used to replace it.

\rSec2[cpp.stringize]{\tcode{\#}操作符}%
\indextext{\#\#0~operator@\tcode{\#}~operator}%
\indextext{stringize|see{\tcode{\#}}}

\pnum
函数式宏替换列表中的每个
\tcode{\#}
预处理符号后都应该紧跟一个形参在替换列表中。

\pnum
\defn{字符字串字面值}是没有前缀的\grammarterm{字符串字面值}。
若替代列表中形参紧跟在预处理符号\tcode{\#}后,
该形参和该\tcode{\#}符号会被一起替换为一个单独的字符字面字面值预处理符号,
该字面值包含了相应实参的预处理符号序列的拼写。
实参的预处理符号序列中出现的每处空白都会替换成一个空格符。
序列的首个预处理符号前方和最末预处理符号后方的空白会被丢弃。
接着,实参中每个预处理符号原本的拼写都会原封不动的
保留在字符字串字面值中,
例外是在处理字符字面值和字符串字面值的拼写时
所有\tcode{"}符(包括分界用的\tcode{"}符)
和
\tcode{\textbackslash}符
前方都会插入一个
\tcode{\textbackslash}符。
如果该替换产生的不是有效的字符字串字面值,
行为是未定义的。
空实参对应的字符字串字面值是\tcode{""}。
\tcode{\#}操作符
和
\tcode{\#\#}操作符
的求值顺序是未规定的。

\rSec2[cpp.concat]{\tcode{\#\#}操作符}%
\indextext{\#\#1 operator@\tcode{\#\#} operator}%
\indextext{concatenation!macro argument|see{\tcode{\#\#}}}

\pnum
不管是哪种风格的宏定义,
替换列表都不应该以预处理符号\tcode{\#\#}开头或结尾。

\pnum
若函数风格宏的替换列表中形参前方或后方存在预处理符号\tcode{\#\#},
则该形参会被替换为对应实参的预处理符号序列;
但若对应实参不含预处理符号,
该形参会被替换为预处理占位符。
\footnote{由于预处理占位符仅是存在于翻译第四阶段的临时物,因此未出现在语法中。}

\pnum
不管是哪种风格的宏调用,
在重新检测替换列表的宏名替换前,
替换列表包含的(不包括实参中包含的)所有的\tcode{\#\#}预处理符号都会被删除,
其前方的预处理符号会与后方的预处理符号进行连接。
预处理占位符需要特殊处理:
连接两个预处理占位符将得到一个单独的预处理占位符;
连接预处理占位符与非预处理占位符将得到该非预处理占位符。
若连接的结果不是有效的预处理符号,
行为是未定义的。
连接产生的预处理符号会进一步参与宏替换。
\tcode{\#\#}操作符之间的求值顺序是未规定的。

\enterexample 下列片段中:

\begin{codeblock}
#define hash_hash # ## #
#define mkstr(a) # a
#define in_between(a) mkstr(a)
#define join(c, d) in_between(c hash_hash d)
char p[] = join(x, y);          // 等价于
                                // char p[] = "x \#\# y";
\end{codeblock}

每个阶段将如此展开:

\begin{codeblock}
join(x, y)
in_between(x hash_hash y)
in_between(x ## y)
mkstr(x ## y)
"x ## y"
\end{codeblock}

换言之,展开\tcode{hash_hash}得到的是新的符号,它由两个相邻的西文井号组成,
但该符号并不是\tcode{\#\#}操作符。\exitexample

\rSec2[cpp.rescan]{Rescanning and further replacement}%
\indextext{macro!rescanning and replacement}%
\indextext{rescanning and replacement|see{macro, rescanning and replacement}}

\pnum
After all parameters in the replacement list have been substituted and \tcode{\#} and \tcode{\#\#} processing has taken
place, all placemarker preprocessing tokens are removed. Then
the resulting preprocessing token sequence is rescanned, along with all
subsequent preprocessing tokens of the source file, for more macro names
to replace.

\pnum
If the name of the macro being replaced is found during this scan of
the replacement list
(not including the rest of the source file's preprocessing tokens),
it is not replaced.
Furthermore,
if any nested replacements encounter the name of the macro being replaced,
it is not replaced.
These nonreplaced macro name preprocessing tokens are no longer available
for further replacement even if they are later (re)examined in contexts
in which that macro name preprocessing token would otherwise have been
replaced.

\pnum
The resulting completely macro-replaced preprocessing token sequence
is not processed as a preprocessing directive even if it resembles one,
but all pragma unary operator expressions within it are then processed as
specified in~\ref{cpp.pragma.op} below.

\rSec2[cpp.scope]{Scope of macro definitions}%
\indextext{macro!scope of definition}%
\indextext{scope!macro definition|see{macro, scope of definition}}

\pnum
A macro definition lasts
(independent of block structure)
until a corresponding
\tcode{\#undef}
directive is encountered or
(if none is encountered)
until the end of the translation unit.
Macro definitions have no significance after translation phase 4.

\pnum
A preprocessing directive of the form

\begin{ncsimplebnf}
\terminal{\# undef} identifier new-line
\indextext{\idxcode{\#undef}}%
\end{ncsimplebnf}

causes the specified identifier no longer to be defined as a macro name.
It is ignored if the specified identifier is not currently defined as
a macro name.

\pnum
\enterexample
The simplest use of this facility is to define a ``manifest constant,''
as in
\begin{codeblock}
#define TABSIZE 100
int table[TABSIZE];
\end{codeblock}
\exitexample

\pnum
\enterexample
The following defines a function-like
macro whose value is the maximum of its arguments.
It has the advantages of working for any compatible types of the arguments
and of generating in-line code without the overhead of function calling.
It has the disadvantages of evaluating one or the other of its arguments
a second time
(including
\indextext{side effects}%
side effects)
and generating more code than a function if invoked several times.
It also cannot have its address taken,
as it has none.

\begin{codeblock}
#define max(a, b) ((a) > (b) ? (a) : (b))
\end{codeblock}

The parentheses ensure that the arguments and
the resulting expression are bound properly.
\exitexample

\pnum
\enterexample
To illustrate the rules for redefinition and reexamination,
the sequence

\begin{codeblock}
#define x       3
#define f(a)    f(x * (a))
#undef  x
#define x       2
#define g       f
#define z       z[0]
#define h       g(~
#define m(a)    a(w)
#define w       0,1
#define t(a)    a
#define p()     int
#define q(x)    x
#define r(x,y)  x ## y
#define str(x)  # x

f(y+1) + f(f(z)) % t(t(g)(0) + t)(1);
g(x+(3,4)-w) | h 5) & m
    (f)^m(m);
p() i[q()] = { q(1), r(2,3), r(4,), r(,5), r(,) };
char c[2][6] = { str(hello), str() };
\end{codeblock}

results in

\begin{codeblock}
f(2 * (y+1)) + f(2 * (f(2 * (z[0])))) % f(2 * (0)) + t(1);
f(2 * (2+(3,4)-0,1)) | f(2 * (~ 5)) & f(2 * (0,1))^m(0,1);
int i[] = { 1, 23, 4, 5, };
char c[2][6] = { "hello", "" };
\end{codeblock}
\exitexample

\pnum
\enterexample
To illustrate the rules for creating character string literals
and concatenating tokens,
the sequence

\begin{codeblock}
#define str(s)      # s
#define xstr(s)     str(s)
#define debug(s, t) printf("x" # s "= %d, x" # t "= %s", @\textbackslash@
               x ## s, x ## t)
#define INCFILE(n)  vers ## n
#define glue(a, b)  a ## b
#define xglue(a, b) glue(a, b)
#define HIGHLOW     "hello"
#define LOW         LOW ", world"

debug(1, 2);
fputs(str(strncmp("abc@\textbackslash@0d", "abc", '@\textbackslash@4')  // this goes away
    == 0) str(: @\atsign\textbackslash@n), s);
#include xstr(INCFILE(2).h)
glue(HIGH, LOW);
xglue(HIGH, LOW)
\end{codeblock}

results in

\begin{codeblock}
printf("x" "1" "= %d, x" "2" "= %s", x1, x2);
fputs("strncmp(@\textbackslash@"abc@\textbackslash\textbackslash@0d@\textbackslash@", @\textbackslash@"abc@\textbackslash@", '@\textbackslash\textbackslash@4') == 0" ": @\atsign\textbackslash@n", s);
#include "vers2.h"    @\textit{(after macro replacement, before file access)}@
"hello";
"hello" ", world"
\end{codeblock}

or, after concatenation of the character string literals,

\begin{codeblock}
printf("x1= %d, x2= %s", x1, x2);
fputs("strncmp(@\textbackslash@"abc@\textbackslash\textbackslash@0d@\textbackslash@", @\textbackslash@"abc@\textbackslash@", '@\textbackslash\textbackslash@4') == 0: @\atsign\textbackslash@n", s);
#include "vers2.h"    @\textit{(after macro replacement, before file access)}@
"hello";
"hello, world"
\end{codeblock}

Space around the
\tcode{\#}
and
\tcode{\#\#}
tokens in the macro definition is optional.
\exitexample

\pnum
\enterexample
To illustrate the rules for placemarker preprocessing tokens, the sequence

\begin{codeblock}
#define t(x,y,z) x ## y ## z
int j[] = { t(1,2,3), t(,4,5), t(6,,7), t(8,9,),
  t(10,,), t(,11,), t(,,12), t(,,) };
\end{codeblock}

results in

\begin{codeblock}
int j[] = { 123, 45, 67, 89,
  10, 11, 12, };
\end{codeblock}
\exitexample

\pnum
\enterexample
To demonstrate the redefinition rules,
the following sequence is valid.

\begin{codeblock}
#define OBJ_LIKE      (1-1)
#define OBJ_LIKE      @\tcode{/* white space */ (1-1) /* other */}@
#define FUNC_LIKE(a)   ( a )
#define FUNC_LIKE( a )(     @\tcode{/* note the white space */ \textbackslash}@
                a @\tcode{/* other stuff on this line}@
                  @\tcode{*/}@ )
\end{codeblock}

But the following redefinitions are invalid:

\begin{codeblock}
#define OBJ_LIKE    (0)      // different token sequence
#define OBJ_LIKE    (1 - 1)  // different white space
#define FUNC_LIKE(b) ( a )   // different parameter usage
#define FUNC_LIKE(b) ( b )   // different parameter spelling
\end{codeblock}
\exitexample

\pnum
\enterexample
Finally, to show the variable argument list macro facilities:

\begin{codeblock}
#define debug(...) fprintf(stderr, @\mname{VA_ARGS}@)
#define showlist(...) puts(#@\mname{VA_ARGS}@)
#define report(test, ...) ((test) ? puts(#test) : printf(@\mname{VA_ARGS}@))
debug("Flag");
debug("X = %d\n", x);
showlist(The first, second, and third items.);
report(x>y, "x is %d but y is %d", x, y);
\end{codeblock}

results in

\begin{codeblock}
fprintf(stderr, "Flag");
fprintf(stderr, "X = %d\n", x);
puts("The first, second, and third items.");
((x>y) ? puts("x>y") : printf("x is %d but y is %d", x, y));
  
\end{codeblock}
\exitexample
\indextext{macro!replacement|)}

\rSec1[cpp.line]{Line control}%
\indextext{preprocessing directives!line control}%
\indextext{\idxcode{\#line}|see{preprocessing directives, line control}}

\pnum
The string literal of a
\tcode{\#line}
directive, if present,
shall be a character string literal.

\pnum
The
\term{line number}
of the current source line is one greater than
the number of new-line characters read or introduced
in translation phase 1~(\ref{lex.phases})
while processing the source file to the current token.

\pnum
A preprocessing directive of the form

\begin{ncsimplebnf}
\terminal{\# line} digit-sequence new-line
\end{ncsimplebnf}

causes the implementation to behave as if
the following sequence of source lines begins with a
source line that has a line number as specified
by the digit sequence (interpreted as a decimal integer).
If the digit sequence specifies zero
or a number greater than 2147483647,
the behavior is undefined.

\pnum
A preprocessing directive of the form

\begin{ncsimplebnf}
\terminal{\# line} digit-sequence \terminal{"} s-char-sequence\opt{} \terminal{"} new-line
\end{ncsimplebnf}

sets the presumed line number similarly and changes the
presumed name of the source file to be the contents
of the character string literal.

\pnum
A preprocessing directive of the form

\begin{ncsimplebnf}
\terminal{\# line} pp-tokens new-line
\end{ncsimplebnf}

(that does not match one of the two previous forms)
is permitted.
The preprocessing tokens after
\tcode{line}
on the directive are processed just as in normal text
(each identifier currently defined as a macro name is replaced by its
replacement list of preprocessing tokens).
If the directive resulting after all replacements does not match
one of the two previous forms, the behavior is undefined;
otherwise, the result is processed as appropriate.

\rSec1[cpp.error]{Error directive}%
\indextext{preprocessing directives!error}%
\indextext{\idxcode{\#error}|see{preprocessing directives, error}}

\pnum
A preprocessing directive of the form

\begin{ncsimplebnf}
\terminal{\# error} pp-tokens\opt new-line
\end{ncsimplebnf}

causes the implementation to produce
a diagnostic message that includes the specified sequence of preprocessing tokens,
and renders the program ill-formed.

\rSec1[cpp.pragma]{Pragma directive}%
\indextext{preprocessing directives!pragma}%
\indextext{\idxcode{\#pragma}|see{preprocessing directives, pragma}}

\pnum
A preprocessing directive of the form

\begin{ncsimplebnf}
\terminal{\# pragma} pp-tokens\opt new-line
\end{ncsimplebnf}

causes the implementation to behave
in an \impldef{\#pragma@\tcode{\#pragma}} manner.
The behavior might cause translation to fail or cause the translator or
the resulting program to behave in a non-conforming manner.
Any pragma that is not recognized by the implementation is ignored.

\rSec1[cpp.null]{Null directive}%
\indextext{preprocessing directives!null}

\pnum
A preprocessing directive of the form

\begin{ncsimplebnf}
\terminal{\#} new-line
\end{ncsimplebnf}

has no effect.

\rSec1[cpp.predefined]{Predefined macro names}
\indextext{macro!predefined}%
\indextext{name!predefined macro|see{macro, predefined}}

\pnum
The following macro names shall be defined by the implementation:

\begin{description}

\indextext{\idxcode{\unun cplusplus}}%
\item \tcode{\xname{cplusplus}}\\
The name \tcode{\,\xname{cplusplus}} is defined
to the value
\tcode{\cppver}
when
compiling a \Cpp translation unit.\footnote{It is intended that future
versions of this standard will
replace the value of this macro with a greater value.
Non-conforming compilers should use a value with at most
five decimal digits.}

\indextext{\xname{DATE}@\mname{DATE}}%
\item \mname{DATE}\\
The date of translation of the source file:
a character string literal of the form
\tcode{"Mmm~dd~yyyy"},
where the names of the months are the same as those generated
by the
\tcode{asctime}
function,
and the first character of
\tcode{dd}
is a space character if the value is less than 10.
If the date of translation is not available,
an \impldef{text of \mname{DATE} when date of translation is not available} valid date
shall be supplied.

\indextext{\xname{FILE}@\mname{FILE}}%
\item \mname{FILE}\\
The presumed name of the current source file (a character string
literal).%
\footnote{The presumed source file name can be changed by the \tcode{\#line} directive.}

\indextext{\xname{LINE}@\mname{LINE}}%
\item \mname{LINE}\\
The presumed line number (within the current source file) of the current source line
(an integer literal).%
\footnote{The presumed line number can be changed by the \tcode{\#line} directive.}

\indextext{\xname{STDC_HOSTED}@\mname{STDC_HOSTED}}%
\indextext{\xname{STDC_HOSTED}@\mname{STDC_HOSTED}!implementation-defined}%
\item \mname{STDC_HOSTED}\\
The integer literal \tcode{1} if the implementation is a hosted
implementation or the integer literal \tcode{0} if it is not.

\indextext{\xname{TIME}@\mname{TIME}}%
\item \mname{TIME}\\
The time of translation of the source file:
a character string literal of the form
\tcode{"hh:mm:ss"}
as in the time generated by the
\tcode{asctime}
function.
If the time of translation is not available,
an \impldef{text of \mname{TIME} when time of translation is not available} valid time shall be supplied.
\end{description}

\pnum
The following macro names are conditionally defined by the implementation:

\begin{description}
  \indextext{\xname{STDC}@\mname{STDC}}%
  \indextext{\xname{STDC}@\mname{STDC}!implementation-defined}%
\item \mname{STDC}\\
Whether \mname{STDC} is predefined and if so, what its value is,
are \impldef{definition and meaning of \mname{STDC}}.

\indextext{\xname{STDC_MB_MIGHT_NEQ_WC}@\mname{STDC_MB_MIGHT_NEQ_WC}}%
\indextext{\xname{STDC_MB_MIGHT_NEQ_WC}@\mname{STDC_MB_MIGHT_NEQ_WC}!implementation-defined}%
\item \mname{STDC_MB_MIGHT_NEQ_WC}\\
The integer literal \tcode{1}, intended to indicate that, in the encoding for
\tcode{wchar_t}, a member of the basic character set need not have a code value equal to
its value when used as the lone character in an ordinary character literal.

\indextext{\xname{STDC_VERSION}@\mname{STDC_VERSION}}%
\indextext{\xname{STDC_VERSION}@\mname{STDC_VERSION}!implementation-defined}%
\item \mname{STDC_VERSION}\\
Whether \mname{STDC_VERSION} is predefined and if so, what its value is,
are \impldef{definition and meaning of \mname{STDC_VERSION}}.

\indextext{\xname{STDC_ISO_10646}@\mname{STDC_ISO_10646}}%
\indextext{\xname{STDC_ISO_10646}@\mname{STDC_ISO_10646}!implementation-defined}%
\item \mname{STDC_ISO_10646}\\
An integer literal of the form \tcode{yyyymmL} (for example,
\tcode{199712L}).
If this symbol is defined, then every character in the Unicode required set, when
stored in an object of type \tcode{wchar_t}, has the same value as the short identifier
of that character. The \defn{Unicode required set} consists of all
the characters that are defined by ISO/IEC 10646, along with
all amendments and technical corrigenda as of the specified year and month.

\indextext{\xname{STDCPP_STRICT_POINTER_SAFETY}@\mname{STDCPP_STRICT_POINTER_SAFETY}}%
\indextext{\xname{STDCPP_STRICT_POINTER_SAFETY}@\mname{STDCPP_STRICT_POINTER_SAFETY}!implementation-defined}%
\item \mname{STDCPP_STRICT_POINTER_SAFETY}\\
Defined, and has the value integer literal 1, if and only if the implementation
has strict pointer safety~(\ref{basic.stc.dynamic.safety}).

\indextext{\xname{STDCPP_THREADS}@\mname{STDCPP_THREADS}}%
\indextext{\xname{STDCPP_THREADS}@\mname{STDCPP_THREADS}!implementation-defined}%
\item \mname{STDCPP_THREADS}\\
Defined, and has the value integer literal 1, if and only if a program
can have more than one thread of execution~(\ref{intro.multithread}).

\end{description}

\pnum
The values of the predefined macros
(except for
\mname{FILE}
and
\mname{LINE})
remain constant throughout the translation unit.

\pnum
If any of the pre-defined macro names in this subclause,
or the identifier
\tcode{defined},
is the subject of a
\tcode{\#define}
or a
\tcode{\#undef}
preprocessing directive,
the behavior is undefined.
Any other predefined macro names shall begin with a
leading underscore followed by an uppercase letter or a second
underscore.

\rSec1[cpp.pragma.op]{Pragma operator}%
\indextext{macro!pragma operator}%
\indextext{operator!pragma|see{macro, pragma operator}}

A unary operator expression of the form:

\begin{ncbnf}
\terminal{_Pragma} \terminal{(} string-literal \terminal{)}
\end{ncbnf}

is processed as follows: The string literal is \term{destringized}
by deleting the \tcode{L} prefix, if present, deleting the leading and trailing
double-quotes, replacing each escape sequence \tcode{\textbackslash"} by a double-quote, and
replacing each escape sequence \tcode{\textbackslash\textbackslash} by a single
backslash. The resulting sequence of characters is processed through translation phase 3
to produce preprocessing tokens that are executed as if they were the
\grammarterm{pp-tokens} in a pragma directive. The original four preprocessing
tokens in the unary operator expression are removed.

\enterexample

\begin{codeblock}
#pragma listing on "..\listing.dir"
\end{codeblock}

can also be expressed as:

\begin{codeblock}
_Pragma ( "listing on \"..\\listing.dir\"" )
\end{codeblock}

The latter form is processed in the same way whether it appears literally
as shown, or results from macro replacement, as in:

\begin{codeblock}
#define LISTING(x) PRAGMA(listing on #x)
#define PRAGMA(x) _Pragma(#x)

LISTING( ..\listing.dir )
\end{codeblock}

\exitexample%
\indextext{preprocessing directives|)}
